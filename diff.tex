%!TEX TS-program = xelatex
%DIF LATEXDIFF DIFFERENCE FILE
%DIF DEL full.tex   Sun Aug 21 11:34:08 2022
%DIF ADD main.tex   Sun Aug 21 11:28:54 2022
% @Author: WU Zihan
% @Date:   2018-02-28 19:42:22
% @Last Modified by:   WU Zihan
%DIF 5c5
%DIF < % @Last Modified time: 2022-08-21 11:39:43
%DIF -------
% @Last Modified time: 2022-08-21 11:25:48 %DIF > 
%DIF -------
\PassOptionsToPackage{quiet}{fontspec}
\documentclass[UTF8]{ctexart}
\usepackage{graphicx} 
\usepackage[namelimits]{amsmath} %数学公式
\usepackage{amssymb}             %数学公式
\usepackage{amsfonts}            %数学字体
\usepackage{mathrsfs}            %数学花体
\usepackage{enumerate}
\usepackage{amsmath,bm}
\usepackage{amsthm}
\usepackage{listings}
\usepackage{appendix}
\usepackage{silence}
\WarningsOff[latexfont]
\newenvironment{mymathfrac}[2]%
{\hbox{$#1$}\! \left/ \! \hbox{$#2$}\right.}

\lstset{breaklines}%自动将长的代码行换行排版
\lstset{extendedchars=false}%解决代码跨页时,章节标题,页眉等汉字不显示的问题

% \newtheorem{Theorem Femat}{Lemma}
\theoremstyle{definition}
\newtheorem{definition}{Definition}
\newtheorem*{assumption}{Assumption}
\newtheorem*{corollary}{Corollary}
\theoremstyle{plain}
\newtheorem{theorem}{Theorem}
\newtheorem{lemma}{Lemma}
% \newtheorem{assumption}{Assumption}
\title{Proof of safety}
\author{Zihan}
\date{\today}
%DIF PREAMBLE EXTENSION ADDED BY LATEXDIFF
%DIF UNDERLINE PREAMBLE %DIF PREAMBLE
\RequirePackage[normalem]{ulem} %DIF PREAMBLE
\RequirePackage{color}\definecolor{RED}{rgb}{1,0,0}\definecolor{BLUE}{rgb}{0,0,1} %DIF PREAMBLE
\providecommand{\DIFadd}[1]{{\protect\color{blue}\uwave{#1}}} %DIF PREAMBLE
\providecommand{\DIFdel}[1]{{\protect\color{red}\sout{#1}}}                      %DIF PREAMBLE
%DIF SAFE PREAMBLE %DIF PREAMBLE
\providecommand{\DIFaddbegin}{} %DIF PREAMBLE
\providecommand{\DIFaddend}{} %DIF PREAMBLE
\providecommand{\DIFdelbegin}{} %DIF PREAMBLE
\providecommand{\DIFdelend}{} %DIF PREAMBLE
\providecommand{\DIFmodbegin}{} %DIF PREAMBLE
\providecommand{\DIFmodend}{} %DIF PREAMBLE
%DIF FLOATSAFE PREAMBLE %DIF PREAMBLE
\providecommand{\DIFaddFL}[1]{\DIFadd{#1}} %DIF PREAMBLE
\providecommand{\DIFdelFL}[1]{\DIFdel{#1}} %DIF PREAMBLE
\providecommand{\DIFaddbeginFL}{} %DIF PREAMBLE
\providecommand{\DIFaddendFL}{} %DIF PREAMBLE
\providecommand{\DIFdelbeginFL}{} %DIF PREAMBLE
\providecommand{\DIFdelendFL}{} %DIF PREAMBLE
\newcommand{\DIFscaledelfig}{0.5}
%DIF HIGHLIGHTGRAPHICS PREAMBLE %DIF PREAMBLE
\RequirePackage{settobox} %DIF PREAMBLE
\RequirePackage{letltxmacro} %DIF PREAMBLE
\newsavebox{\DIFdelgraphicsbox} %DIF PREAMBLE
\newlength{\DIFdelgraphicswidth} %DIF PREAMBLE
\newlength{\DIFdelgraphicsheight} %DIF PREAMBLE
% store original definition of \includegraphics %DIF PREAMBLE
\LetLtxMacro{\DIFOincludegraphics}{\includegraphics} %DIF PREAMBLE
\newcommand{\DIFaddincludegraphics}[2][]{{\color{blue}\fbox{\DIFOincludegraphics[#1]{#2}}}} %DIF PREAMBLE
\newcommand{\DIFdelincludegraphics}[2][]{% %DIF PREAMBLE
\sbox{\DIFdelgraphicsbox}{\DIFOincludegraphics[#1]{#2}}% %DIF PREAMBLE
\settoboxwidth{\DIFdelgraphicswidth}{\DIFdelgraphicsbox} %DIF PREAMBLE
\settoboxtotalheight{\DIFdelgraphicsheight}{\DIFdelgraphicsbox} %DIF PREAMBLE
\scalebox{\DIFscaledelfig}{% %DIF PREAMBLE
\parbox[b]{\DIFdelgraphicswidth}{\usebox{\DIFdelgraphicsbox}\\[-\baselineskip] \rule{\DIFdelgraphicswidth}{0em}}\llap{\resizebox{\DIFdelgraphicswidth}{\DIFdelgraphicsheight}{% %DIF PREAMBLE
\setlength{\unitlength}{\DIFdelgraphicswidth}% %DIF PREAMBLE
\begin{picture}(1,1)% %DIF PREAMBLE
\thicklines\linethickness{2pt} %DIF PREAMBLE
{\color[rgb]{1,0,0}\put(0,0){\framebox(1,1){}}}% %DIF PREAMBLE
{\color[rgb]{1,0,0}\put(0,0){\line( 1,1){1}}}% %DIF PREAMBLE
{\color[rgb]{1,0,0}\put(0,1){\line(1,-1){1}}}% %DIF PREAMBLE
\end{picture}% %DIF PREAMBLE
}\hspace*{3pt}}} %DIF PREAMBLE
} %DIF PREAMBLE
\LetLtxMacro{\DIFOaddbegin}{\DIFaddbegin} %DIF PREAMBLE
\LetLtxMacro{\DIFOaddend}{\DIFaddend} %DIF PREAMBLE
\LetLtxMacro{\DIFOdelbegin}{\DIFdelbegin} %DIF PREAMBLE
\LetLtxMacro{\DIFOdelend}{\DIFdelend} %DIF PREAMBLE
\DeclareRobustCommand{\DIFaddbegin}{\DIFOaddbegin \let\includegraphics\DIFaddincludegraphics} %DIF PREAMBLE
\DeclareRobustCommand{\DIFaddend}{\DIFOaddend \let\includegraphics\DIFOincludegraphics} %DIF PREAMBLE
\DeclareRobustCommand{\DIFdelbegin}{\DIFOdelbegin \let\includegraphics\DIFdelincludegraphics} %DIF PREAMBLE
\DeclareRobustCommand{\DIFdelend}{\DIFOaddend \let\includegraphics\DIFOincludegraphics} %DIF PREAMBLE
\LetLtxMacro{\DIFOaddbeginFL}{\DIFaddbeginFL} %DIF PREAMBLE
\LetLtxMacro{\DIFOaddendFL}{\DIFaddendFL} %DIF PREAMBLE
\LetLtxMacro{\DIFOdelbeginFL}{\DIFdelbeginFL} %DIF PREAMBLE
\LetLtxMacro{\DIFOdelendFL}{\DIFdelendFL} %DIF PREAMBLE
\DeclareRobustCommand{\DIFaddbeginFL}{\DIFOaddbeginFL \let\includegraphics\DIFaddincludegraphics} %DIF PREAMBLE
\DeclareRobustCommand{\DIFaddendFL}{\DIFOaddendFL \let\includegraphics\DIFOincludegraphics} %DIF PREAMBLE
\DeclareRobustCommand{\DIFdelbeginFL}{\DIFOdelbeginFL \let\includegraphics\DIFdelincludegraphics} %DIF PREAMBLE
\DeclareRobustCommand{\DIFdelendFL}{\DIFOaddendFL \let\includegraphics\DIFOincludegraphics} %DIF PREAMBLE
%DIF COLORLISTINGS PREAMBLE %DIF PREAMBLE
\RequirePackage{listings} %DIF PREAMBLE
\RequirePackage{color} %DIF PREAMBLE
\lstdefinelanguage{DIFcode}{ %DIF PREAMBLE
%DIF DIFCODE_UNDERLINE %DIF PREAMBLE
  moredelim=[il][\color{red}\sout]{\%DIF\ <\ }, %DIF PREAMBLE
  moredelim=[il][\color{blue}\uwave]{\%DIF\ >\ } %DIF PREAMBLE
} %DIF PREAMBLE
\lstdefinestyle{DIFverbatimstyle}{ %DIF PREAMBLE
	language=DIFcode, %DIF PREAMBLE
	basicstyle=\ttfamily, %DIF PREAMBLE
	columns=fullflexible, %DIF PREAMBLE
	keepspaces=true %DIF PREAMBLE
} %DIF PREAMBLE
\lstnewenvironment{DIFverbatim}{\lstset{style=DIFverbatimstyle}}{} %DIF PREAMBLE
\lstnewenvironment{DIFverbatim*}{\lstset{style=DIFverbatimstyle,showspaces=true}}{} %DIF PREAMBLE
%DIF END PREAMBLE EXTENSION ADDED BY LATEXDIFF

\begin{document}
\maketitle

\section{Preliminary}
我们考察一种情形,攻击者的算力与诚实者算力之比为$q \in [0,1)$. 但是在此之前,我们先对算力与权重的关系进行一些分析。

\DIFdelbegin \subsection{\DIFdel{一些定义和假设}}
%DIFAUXCMD
\addtocounter{subsection}{-1}%DIFAUXCMD
\DIFdelend %DIF >  % \subsection{一些定义和假设}
\begin{definition}
    Let $B\in G$ be a block. 称 $B$为一个 $2^k$-block, 若其所在的 Sibling Group 有 $2^k$ 个区块. 
所有的$2^k$-block构成的集合记为$\mathcal{B}_k$.
\end{definition}


% 称block $B$为一个 $2^k$-block, 若其所在的 Sibling Group 有 $2^k$ 个区块. 
% 所有的$2^k$-block构成的集合记为$\mathcal{B}_k$.

% 记$a_l$为

% \subsection{期望Hash次数}
\begin{assumption}
    和其他文献一样, 我们把生成同一种区块的Hash过程视作一个Poisson过程. 具体来说, 若在$\mathcal{B}_k$上进行$N$次Hash, 记此时生成的$2^k$-block
    数目为$a_k(N)$, 则$a_k(N)$是一个强度为$2^kp$的Poisson过程, 其中$p$是Mining Difficulty.
\end{assumption}
\begin{corollary}
    由Poisson过程, \DIFaddbegin \DIFadd{期望}\DIFaddend ${\rm E}(a_k(N)) = 2^k p N$\DIFdelbegin \DIFdel{.
}\DIFdelend \DIFaddbegin \DIFadd{, 方差${\rm Var}(a_k(N))= 2^k p N.$
}\DIFaddend \end{corollary}
生成的每个$2^k$-块的权重$W_k$设定为$W_k = \cfrac{1}{2^k p}$, 且记诚实者算力为$v_h$.


\section{权重与Hash次数的关系}
对于Hash次数和所生成的权重, 我们有以下定理:

\begin{theorem}
    给定Hash次数$N$, 所生成的权重$W(N)$, 作为一个随机变量, 其期望与具体生成区块的种类无关, 且有
    \[ {\rm {E}}(W(N)) = N \]
    \DIFaddbegin \[ {\DIFadd{\rm }{\DIFadd{Var}}}\DIFadd{(W(N)) \leq \cfrac{N}{p}}\]
    %DIF >  其中$k_{\max} = |\{\mathcal{B}_k\}|.$
    \DIFaddend \label{weight}
\end{theorem}
\begin{proof}
    $N$次的Hash是分配在各类区块上的. 由于$G$是有限大小的, $|\{\mathcal{B}_k\}|\in \mathbb{N}$, 记为$k_{\rm max}$.
    于是有
    \[N = N\DIFdelbegin \DIFdel{_1 }\DIFdelend \DIFaddbegin \DIFadd{_0 }\DIFaddend + N\DIFdelbegin \DIFdel{_2 }\DIFdelend \DIFaddbegin \DIFadd{_1 }\DIFaddend + \dots + N_{k_{\rm max}},\] 其中 \DIFdelbegin \DIFdel{$N_k (1 \le k \le k_{\rm max})$ }\DIFdelend \DIFaddbegin \DIFadd{$N_k (0 \le k \le k_{\rm max})$ }\DIFaddend 表示在$2^k$-block上($\mathcal{B}_k$)上的Hash次数.
    Analogously, 权重$W(N)$ 亦有
    \[W(N) = W\DIFdelbegin \DIFdel{_1}\DIFdelend \DIFaddbegin \DIFadd{_0}\DIFaddend (N\DIFdelbegin \DIFdel{_1}\DIFdelend \DIFaddbegin \DIFadd{_0}\DIFaddend ) + W\DIFdelbegin \DIFdel{_2}\DIFdelend \DIFaddbegin \DIFadd{_1}\DIFaddend (N\DIFdelbegin \DIFdel{_2}\DIFdelend \DIFaddbegin \DIFadd{_1}\DIFaddend ) + \dots + W_{k_{\rm max}}(N_{k_{\rm max}}).\]
    考虑$\mathcal{B}_k$上的\DIFdelbegin \DIFdel{$W_k(N)$}\DIFdelend \DIFaddbegin \DIFadd{$W_k(N_k)$}\DIFaddend , 由于所有生成的块, 根据算法, 都为首 sibling group 贡献了权重, 所以$\mathcal{B}_k$上所提供的权重
    \[W_k(N\DIFaddbegin \DIFadd{_k}\DIFaddend ) = a_k(N\DIFaddbegin \DIFadd{_k}\DIFaddend )W_k = \DIFdelbegin %DIFDELCMD < \cfrac{a_k(N)}{2^k p}%%%
\DIFdelend \DIFaddbegin \cfrac{a_k(N_k)}{2^k p}\DIFaddend ,\]所以有
    \[{\rm E}(W_k(N\DIFaddbegin \DIFadd{_k}\DIFaddend ))=\DIFdelbegin %DIFDELCMD < \cfrac{{\rm E}(a_k(N))}{2^k p} %%%
\DIFdelend \DIFaddbegin \cfrac{{\rm E}(a_k(N_k))}{2^k p} \DIFaddend = \DIFdelbegin %DIFDELCMD < \cfrac{2^k p N}{2^k p} %%%
\DIFdelend \DIFaddbegin \cfrac{2^k p N_k}{2^k p} \DIFaddend = N\DIFaddbegin \DIFadd{_k}\DIFaddend .\]
    \DIFaddbegin \DIFadd{对方差也有
    }\[{\DIFadd{\rm Var}}\DIFadd{(W_k(N_k))=\cfrac{{\rm Var}(a_k(N_k))}{(2^k p)^2} = \cfrac{N_k}{2^k p}.}\]
    \DIFaddend 所以总权重期望有
    \DIFdelbegin \[ {\DIFdel{\rm }{\DIFdel{E}}}\DIFdel{(W(N)) =}{\DIFdel{\rm }{\DIFdel{E}}}\DIFdel{\left(\sum_{1 \le k \le k_{\rm max}} W_k(N_k)\right) =\sum_{1 \le k \le k_{\rm max}} }{\DIFdel{\rm }{\DIFdel{E}}}\DIFdel{(W_k(N_k))}\]%DIFAUXCMD
\DIFdelend \DIFaddbegin \begin{align*}
        {\DIFadd{\rm }{\DIFadd{E}}}\DIFadd{(W(N)) }&\DIFadd{=}{\DIFadd{\rm }{\DIFadd{E}}}\DIFadd{\left(\sum_{0 \le k \le k_{\rm max}} W_k(N_k)\right) =\sum_{0 \le k \le k_{\rm max}} }{\DIFadd{\rm }{\DIFadd{E}}}\DIFadd{(W_k(N_k))
        }\\ &\DIFadd{= \sum_{0 \le k \le k_{\rm max}}N_k = N
    }\end{align*}\DIFaddend 
    \DIFdelbegin \[\DIFdel{= \sum_{1 \le k \le k_{\rm max}}N_k = N}\]%DIFAUXCMD
\DIFdelend \DIFaddbegin \DIFadd{而且由于$\{W_k(N_K)\}$相互独立,总权重方差
    }\begin{align*}
        {\DIFadd{\rm }{\DIFadd{Var}}}\DIFadd{(W(N)) }&\DIFadd{=}{\DIFadd{\rm }{\DIFadd{Var}}}\DIFadd{\left(\sum_{0 \le k \le k_{\rm max}} W_k(N_k)\right) =\sum_{0 \le k \le k_{\rm max}} }{\DIFadd{\rm }{\DIFadd{Var}}}\DIFadd{(W_k(N_k))
        }\\ &\DIFadd{= \sum_{0 \le k \le k_{\rm max}}\cfrac{N_k}{2^k p} \leq \sum_{0 \le k \le k_{\rm max}}\cfrac{N_k}{p} = \cfrac{N}{p}.
    }\end{align*}\DIFaddend 
\end{proof}

\section{诚实者和攻击者}
记事件攻击者篡夺成功为事件$F$. 攻击者若想篡夺主视图, 其首sibling 权重应当大于主视图的首sibling 权重. 
即\[F=\{W_a>W_h+W_0\}.\]
此时, 我们有如下的安全定理:
\begin{theorem}
    $\forall \epsilon>0, \exists T>0$, s.t. $\forall t>T$,
    \[{\rm P}(F) <\epsilon .\]
    \label{safethm}
\end{theorem}
先证明一个引理:
\begin{lemma}
    若记$\Delta = {\rm E}(W_h)-{\rm E}(W_a)$, 则进一步有
\[F\subseteq \{{\rm E}(W_h)-W_h>\cfrac{\Delta}{2}\} \cup \{W_a-{\rm E}(W_a)>\cfrac{\Delta}{2}\}.\]
\label{lem}
\end{lemma}

\begin{proof}
$\forall\omega\notin\{{\rm E}(W_h)-W_h>\cfrac{\Delta}{2}\} \cup \{W_a-{\rm E}(W_a)>\cfrac{\Delta}{2}\}$, 有
\[\omega\in \{{\rm E}(W_h)-W_h\le\cfrac{\Delta}{2}\} \cap \{W_a-{\rm E}(W_a)\le\cfrac{\Delta}{2}\}.\] 此时
\[{\rm E}(W_h)-W_h+W_a-{\rm E}(W_a)\le\Delta,\]即
\[W_a-W_h \le \Delta - {\rm E}(W_h)-{\rm E}(W_a)= 0 \]
于是\[\omega \notin F,\] 也即\[F \subseteq \{{\rm E}(W_h)-W_h>\cfrac{\Delta}{2}\} \cup \{W_a-{\rm E}(W_a)>\cfrac{\Delta}{2}\}.\]
\end{proof}

下面我们证明Theorem \ref{safethm}:
\begin{proof}
    由Lemma \ref{lem},
    \begin{align*}
        {\rm P}(A) \le {\rm P}(\{{\rm E}(W_h)-W_h>\cfrac{\Delta}{2}\} \cup \{W_a-{\rm E}(W_a)>\cfrac{\Delta}{2}\}) \\ \le 
    {\rm P}({\rm E}(W_h)-W_h>\cfrac{\Delta}{2})+{\rm P}(W_a-{\rm E}(W_a)>\cfrac{\Delta}{2}) \\ 
    \le {\rm P}(|{\rm E}(W_h)-W_h|>\cfrac{\Delta}{2})+{\rm P}(|W_a-{\rm E}(W_a)|>\cfrac{\Delta}{2}).
    \end{align*}
    由切比雪夫不等式, 
    \begin{align*}
        {\rm P}(\DIFdelbegin \DIFdel{A}\DIFdelend \DIFaddbegin \DIFadd{F}\DIFaddend ) &\le \cfrac{{\rm Var}(W_h)}{(\Delta/2)^2} + \cfrac{{\rm Var}(W_a)}{(\Delta/2)^2} \\ 
        &=  \cfrac{4({\rm Var}(W_h)+{\rm Var}(W_a))}{\Delta^2}
        % &= \cfrac{4({\rm Var}(W_h)+{\rm Var}(W_a))}{\Delta^2}
    \end{align*}
    \DIFdelbegin \DIFdel{而对于$\Delta$, }\DIFdelend 由Theorem \ref{weight}
    \[\Delta = {\rm E}(W_h)-{\rm E}(W_a) = N_h - N_a = v_ht-qv_ht=(1-q)v_ht.\]
    \DIFdelbegin \DIFdel{记$f(t) = \cfrac{4({\rm Var}(W_h)+{\rm Var}(W_a))}{[(1-q)v_ht]^2}$, 
    注意到
    }\DIFdelend \[\DIFdelbegin \DIFdel{f}\DIFdelend \DIFaddbegin {\rm \DIFadd{Var}}\DIFaddend (\DIFdelbegin \DIFdel{t}\DIFdelend \DIFaddbegin \DIFadd{W_h}\DIFaddend ) \DIFdelbegin \DIFdel{\sim }%DIFDELCMD < \cfrac{1}{t^2}%%%
\DIFdel{,}\DIFdelend \DIFaddbegin \DIFadd{+ }{\rm \DIFadd{Var}}\DIFadd{(W_a) = }\cfrac{N_h+N_a}{p} \DIFadd{= }\cfrac{(1+q)v_ht}{p}\DIFadd{.}\DIFaddend \]
    \DIFdelbegin \DIFdel{则显然有
    }\DIFdelend \DIFaddbegin \DIFadd{所以 }\DIFaddend \[\DIFdelbegin \DIFdel{\lim_{t\rightarrow \infty}f}\DIFdelend \DIFaddbegin {\rm \DIFadd{P}}\DIFaddend (\DIFdelbegin \DIFdel{t}\DIFdelend \DIFaddbegin \DIFadd{F}\DIFaddend ) \DIFaddbegin \le \cfrac{4(1+q)v_ht}{p [(1-q)v_ht]^2} \DIFaddend = \DIFdelbegin \DIFdel{0.}\DIFdelend \DIFaddbegin \cfrac{4(1+q)}{p(1-q)^2v_h} \cfrac{1}{t}\DIFadd{. }\DIFaddend \]
    \DIFdelbegin \DIFdel{也即
    $\forall \epsilon>0, \exists T>0$, }\DIFdelend %DIF >  记$f(t) = \cfrac{4(1+q)}{p(1-q)^2} \cfrac{1}{v_ht}$, 
    %DIF >  注意到
    %DIF >  \[f(t) \sim \cfrac{1}{t^2},\]
    %DIF >  则显然有
    %DIF >  \[\lim_{t\rightarrow \infty}f(t) = 0.\]
    %DIF >  也即
    \DIFaddbegin \DIFadd{则
    $\forall \epsilon>0$, select $T = \lceil \cfrac{4(1+q)}{p(1-q)^2} \cfrac{1}{v_h \epsilon} \rceil$, }\DIFaddend s.t. $\forall t>T$, 都有
    \DIFaddbegin
    \[\DIFadd{{\rm P}(F) < f(T) \le \cfrac{4(1+q)}{p(1-q)^2v_h}\bigg/\cfrac{4(1+q)}{p(1-q)^2} \cfrac{1}{v_h \epsilon} = \epsilon}.\]

    \DIFaddend
    % \[\DIFdelbegin \DIFdel{|f}\DIFdelend \DIFaddbegin {\rm \DIFadd{P}}\DIFaddend (\DIFdelbegin \DIFdel{t}\DIFdelend \DIFaddbegin \DIFadd{F}\DIFaddend ) \DIFdelbegin \DIFdel{-0|}\DIFdelend < \DIFaddbegin \DIFadd{f(T) }\le \cfrac{4(1+q)}{p(1-q)^2v_h}\bigg\DIFadd{/}\cfrac{4(1+q)}{p(1-q)^2} \cfrac{1}{v_h \epsilon} \DIFadd{= }\DIFaddend \epsilon.\]\DIFdelbegin %DIFDELCMD < 
%DIFDELCMD <     %%%
\DIFdel{此时
    }\[{\DIFdel{\rm P}}\DIFdel{(F) < \epsilon,}\]%DIFAUXCMD
\DIFdelend 
    % \[{\rm P}(A) \le {\rm P}(\{{\rm E}(W_h)-W_h>\cfrac{\Delta}{2}\} \cup \{W_a-{\rm E}(W_a)>\cfrac{\Delta}{2}\}) \le 
    % {\rm P}((W_h)-W_h>\cfrac{\Delta}{2})+{\rm P}(W_a-{\rm E}(W_a)>\cfrac{\Delta}{2})\]
\end{proof}


% (1-q)v_h t
% 和其他文献一样,我们把生成同一种区块的Hash过程视作一个参数为$p$的Poisson过程,其中$p$也是单次Hash生成区块的概率。
% 则若记$M$为
% 则若记$N$为生成一个区块所进行的Hash次数,$N$作为一个随机变量,满足参数为$p$的几何分布. 即
% \[N \sim G(p).\]
% 对其期望与方差,我们有如下结论:
% \[\text{E}(N)=\frac{1}{p};\]
% \[\text{var}(N) = \cfrac{1-p}{p^2}.\]

% 特别地,
% 对某一个单块,$p = 1/2^m,$ 其中 $m$ 代表 leading zero 的数目.

% 考察某Block $B$, 若其所在的 Sibling Group 有 $2^k$ 个区块,则我们称 $B$ 为一个 $2^k$-block. 

% 由于$2^k$-block 的实际leading zero数目为$m-k$, 所以其对应的 $p$ 有
% \[p = \cfrac{1}{2^{m-k}}.\]

% 若在$\mathcal{B}_k$上Hash了$M_k$次, 记生成$2^k$-block 的数目是$a_k$, 则有
% \[a_k \sim Poisson(p)\]

% 于是,若记$N_k$为生成一个$2^k$-block所进行的Hash次数,我们有
% $$n_k\equiv\text{E}(N_k)=\cfrac{1}{\frac{1}{2^{m-k}}}=2^{m-k}.$$

% 如前所述,我们定义了一个$2^k$-block的权重$W_k$为
% \[W_k=2^{m-k}.\]
% 因此我们可以定义


\end{document}






















