%xetex
% @Author: WU Zihan
% @Date:   2018-02-28 19:42:22
% @Last Modified by:   WU Zihan
% @Last Modified time: 2022-08-14 00:20:45
\documentclass[UTF8]{ctexart}
\usepackage{graphicx} 
\usepackage[namelimits]{amsmath} %数学公式
\usepackage{amssymb}             %数学公式
\usepackage{amsfonts}            %数学字体
\usepackage{mathrsfs}            %数学花体
\usepackage{enumerate}
\usepackage{amsmath,bm}
\usepackage{amsthm}
\usepackage{listings}
\usepackage{appendix}
\newenvironment{mymathfrac}[2]%
{\hbox{$#1$}\! \left/ \! \hbox{$#2$}\right.}

\lstset{breaklines}%自动将长的代码行换行排版
\lstset{extendedchars=false}%解决代码跨页时,章节标题,页眉等汉字不显示的问题

\newtheorem{Theorem Femat}{Lemma}

\title{Proof of safety}
\author{Zihan}
\date{\today}
\begin{document}
\maketitle

\section{Preliminary}
我们考察一种情形,攻击者的算力与诚实者算力之比为$q \in [0,1)$. 但是在此之前,我们先对算力与权重的关系进行一些分析。

\paragraph{期望Hash次数}

和其他文献一样,我们把生成同质区块数目随Hash次数的变化视作一个参数为$p$的Poisson过程,其中$p$也是单次Hash生成区块的概率。
则若记$N$为生成一个区块所进行的Hash次数,$N$作为一个随机变量,满足参数为$p$的几何分布. 即
\[N \sim G(p).\]
对其期望与方差,我们有如下结论:
\[\text{E}(N)=\frac{1}{p};\]
\[\text{var}(N) = \cfrac{1-p}{p^2}.\]

特别地,对某一个单块,$p = 1/2^m,$ 其中 $m$ 代表 leading zero 的数目.

考察某Block $B$, 若其所在的 Sibling Group 有 $2^k$ 个区块,则我们称 $B$ 为一个 $2^k$-block. 由于$2^k$-block 的
实际leading zero数目为$m-k$, 所以其对应的 $p$ 有
\[p = \cfrac{1}{2^{m-k}}.\]
于是,若记$N_k$为生成一个$2^k$-block所进行的Hash次数,我们有
$$n_k\equiv\text{E}(N_k)=\cfrac{1}{\frac{1}{2^{m-k}}}=2^{m-k}.$$

如前所述,我们定义了一个$2^k$-block的权重$W_k$为
\[W_k=2^{m-k}.\]
因此我们可以定义

\end{document}






















