%!TEX TS-program = xelatex
% @Author: WU Zihan
% @Date:   2018-02-28 19:42:22
% @Last Modified by:   WU Zihan
% @Last Modified time: 2022-08-18 17:10:36
\PassOptionsToPackage{quiet}{fontspec}
\documentclass[UTF8]{ctexart}
\usepackage{graphicx} 
\usepackage[namelimits]{amsmath} %数学公式
\usepackage{amssymb}             %数学公式
\usepackage{amsfonts}            %数学字体
\usepackage{mathrsfs}            %数学花体
\usepackage{enumerate}
\usepackage{amsmath,bm}
\usepackage{amsthm}
\usepackage{listings}
\usepackage{appendix}
\usepackage{silence}
\WarningsOff[latexfont]
\newenvironment{mymathfrac}[2]%
{\hbox{$#1$}\! \left/ \! \hbox{$#2$}\right.}

\lstset{breaklines}%自动将长的代码行换行排版
\lstset{extendedchars=false}%解决代码跨页时,章节标题,页眉等汉字不显示的问题

% \newtheorem{Theorem Femat}{Lemma}
\theoremstyle{definition}
\newtheorem{definition}{Definition}
\newtheorem*{assumption}{Assumption}
\newtheorem*{corollary}{Corollary}
\theoremstyle{plain}
\newtheorem{theorem}{Theorem}
\newtheorem{lemma}{Lemma}
% \newtheorem{assumption}{Assumption}
\title{Proof of safety}
\author{Zihan}
\date{\today}
\begin{document}
\maketitle

\section{Preliminary}
我们考察一种情形,攻击者的算力与诚实者算力之比为$q \in [0,1)$. 但是在此之前,我们先对算力与权重的关系进行一些分析。

\subsection{一些定义和假设}
\begin{definition}
    Let $B\in G$ be a block. 称 $B$为一个 $2^k$-block, 若其所在的 Sibling Group 有 $2^k$ 个区块. 
所有的$2^k$-block构成的集合记为$\mathcal{B}_k$.
\end{definition}


% 称block $B$为一个 $2^k$-block, 若其所在的 Sibling Group 有 $2^k$ 个区块. 
% 所有的$2^k$-block构成的集合记为$\mathcal{B}_k$.

% 记$a_l$为

% \subsection{期望Hash次数}
\begin{assumption}
    和其他文献一样, 我们把生成同一种区块的Hash过程视作一个Poisson过程. 具体来说, 若在$\mathcal{B}_k$上进行$N$次Hash, 记此时生成的$2^k$-block
    数目为$a_k(N)$, 则$a_k(N)$是一个强度为$2^kp$的Poisson过程, 其中$p$是Mining Difficulty.
\end{assumption}
\begin{corollary}
    由Poisson过程, ${\rm E}(a_k(N)) = 2^k p N$.
\end{corollary}
生成的每个$2^k$-块的权重$W_k$设定为$W_k = \cfrac{1}{2^k p}$, 且记诚实者算力为$v_h$.


\section{权重与Hash次数的关系}
对于Hash次数和所生成的权重, 我们有以下定理:

\begin{theorem}
    给定Hash次数$N$, 所生成的权重$W(N)$, 作为一个随机变量, 其期望与具体生成区块的种类无关, 且有
    \[ {\rm {E}}(W(N)) = N \]
\end{theorem}
\begin{proof}
    $N$次的Hash是分配在各类区块上的. 由于$G$是有限大小的, $|\{\mathcal{B}_k\}|\in \mathbb{N}$, 记为$k_{\rm max}$.
    于是有
    \[N = N_1 + N_2 + \dots + N_{k_{\rm max}},\] 其中 $N_k (1 \le k \le k_{\rm max})$ 表示在$2^k$-block上($\mathcal{B}_k$)上的Hash次数.
    Analogously, 权重$W(N)$ 亦有
    \[W(N) = W_1(N_1) + W_2(N_2) + \dots + W_{k_{\rm max}}(N_{k_{\rm max}}).\]
    考虑$\mathcal{B}_k$上的$W_k(N)$, 由于所有生成的块, 根据算法, 都为首 sibling group 贡献了权重, 所以$\mathcal{B}_k$上所提供的权重
    \[W_k(N) = a_k(N)W_k = \cfrac{a_k(N)}{2^k p},\]所以有
    \[{\rm E}(W_k(N))=\cfrac{{\rm E}(a_k(N))}{2^k p} = \cfrac{2^k p N}{2^k p} = N.\]
    所以总权重期望有
    \[ {\rm {E}}(W(N)) ={\rm {E}}\left(\sum_{1 \le k \le k_{\rm max}} W_k(N_k)\right) =\sum_{1 \le k \le k_{\rm max}} {\rm {E}}(W_k(N_k))\]
    \[= \sum_{1 \le k \le k_{\rm max}}N_k = N\]
\end{proof}

\section{诚实者和攻击者}
记事件攻击者篡夺成功为事件$F$. 攻击者若想篡夺主视图, 其首sibling 权重应当大于主视图的首sibling 权重. 
即\[F=\{W_a>W_h+W_0\}.\]
此时, 我们有如下的安全定理:
\begin{theorem}
    $\forall \epsilon>0, \exists T>0$, s.t. $\forall t>T$,
    \[{\rm P}(F) <\epsilon .\]
    \label{safethm}
\end{theorem}
先证明一个引理:
\begin{lemma}
    若记$\Delta = {\rm E}(W_h)-{\rm E}(W_a)$, 则进一步有
\[F\subseteq \{{\rm E}(W_h)-W_h>\cfrac{\Delta}{2}\} \cup \{W_a-{\rm E}(W_a)>\cfrac{\Delta}{2}\}.\]
\label{lem}
\end{lemma}

\begin{proof}
$\forall\omega\notin\{{\rm E}(W_h)-W_h>\cfrac{\Delta}{2}\} \cup \{W_a-{\rm E}(W_a)>\cfrac{\Delta}{2}\}$, 有
\[\omega\in \{{\rm E}(W_h)-W_h\le\cfrac{\Delta}{2}\} \cap \{W_a-{\rm E}(W_a)\le\cfrac{\Delta}{2}\}.\] 此时
\[{\rm E}(W_h)-W_h+W_a-{\rm E}(W_a)\le\Delta,\]即
\[W_a-W_h \le \Delta - {\rm E}(W_h)-{\rm E}(W_a)= 0 \]
于是\[\omega \notin F,\] 也即\[F \subseteq \{{\rm E}(W_h)-W_h>\cfrac{\Delta}{2}\} \cup \{W_a-{\rm E}(W_a)>\cfrac{\Delta}{2}\}.\]
\end{proof}

下面我们证明Theorem \ref{safethm}:
\begin{proof}
    由Lemma \ref{lem}
\end{proof}


% (1-q)v_h t
% 和其他文献一样,我们把生成同一种区块的Hash过程视作一个参数为$p$的Poisson过程,其中$p$也是单次Hash生成区块的概率。
% 则若记$M$为
% 则若记$N$为生成一个区块所进行的Hash次数,$N$作为一个随机变量,满足参数为$p$的几何分布. 即
% \[N \sim G(p).\]
% 对其期望与方差,我们有如下结论:
% \[\text{E}(N)=\frac{1}{p};\]
% \[\text{var}(N) = \cfrac{1-p}{p^2}.\]

% 特别地,
% 对某一个单块,$p = 1/2^m,$ 其中 $m$ 代表 leading zero 的数目.

% 考察某Block $B$, 若其所在的 Sibling Group 有 $2^k$ 个区块,则我们称 $B$ 为一个 $2^k$-block. 

% 由于$2^k$-block 的实际leading zero数目为$m-k$, 所以其对应的 $p$ 有
% \[p = \cfrac{1}{2^{m-k}}.\]

% 若在$\mathcal{B}_k$上Hash了$M_k$次, 记生成$2^k$-block 的数目是$a_k$, 则有
% \[a_k \sim Poisson(p)\]

% 于是,若记$N_k$为生成一个$2^k$-block所进行的Hash次数,我们有
% $$n_k\equiv\text{E}(N_k)=\cfrac{1}{\frac{1}{2^{m-k}}}=2^{m-k}.$$

% 如前所述,我们定义了一个$2^k$-block的权重$W_k$为
% \[W_k=2^{m-k}.\]
% 因此我们可以定义


\end{document}






















